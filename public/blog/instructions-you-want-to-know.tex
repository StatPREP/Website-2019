\documentclass[]{article}
\usepackage{lmodern}
\usepackage{amssymb,amsmath}
\usepackage{ifxetex,ifluatex}
\usepackage{fixltx2e} % provides \textsubscript
\ifnum 0\ifxetex 1\fi\ifluatex 1\fi=0 % if pdftex
  \usepackage[T1]{fontenc}
  \usepackage[utf8]{inputenc}
\else % if luatex or xelatex
  \ifxetex
    \usepackage{mathspec}
  \else
    \usepackage{fontspec}
  \fi
  \defaultfontfeatures{Ligatures=TeX,Scale=MatchLowercase}
\fi
% use upquote if available, for straight quotes in verbatim environments
\IfFileExists{upquote.sty}{\usepackage{upquote}}{}
% use microtype if available
\IfFileExists{microtype.sty}{%
\usepackage{microtype}
\UseMicrotypeSet[protrusion]{basicmath} % disable protrusion for tt fonts
}{}
\usepackage[margin=1in]{geometry}
\usepackage{hyperref}
\hypersetup{unicode=true,
            pdftitle={You Want to Know: What's the Point?},
            pdfborder={0 0 0},
            breaklinks=true}
\urlstyle{same}  % don't use monospace font for urls
\usepackage{graphicx,grffile}
\makeatletter
\def\maxwidth{\ifdim\Gin@nat@width>\linewidth\linewidth\else\Gin@nat@width\fi}
\def\maxheight{\ifdim\Gin@nat@height>\textheight\textheight\else\Gin@nat@height\fi}
\makeatother
% Scale images if necessary, so that they will not overflow the page
% margins by default, and it is still possible to overwrite the defaults
% using explicit options in \includegraphics[width, height, ...]{}
\setkeys{Gin}{width=\maxwidth,height=\maxheight,keepaspectratio}
\IfFileExists{parskip.sty}{%
\usepackage{parskip}
}{% else
\setlength{\parindent}{0pt}
\setlength{\parskip}{6pt plus 2pt minus 1pt}
}
\setlength{\emergencystretch}{3em}  % prevent overfull lines
\providecommand{\tightlist}{%
  \setlength{\itemsep}{0pt}\setlength{\parskip}{0pt}}
\setcounter{secnumdepth}{0}
% Redefines (sub)paragraphs to behave more like sections
\ifx\paragraph\undefined\else
\let\oldparagraph\paragraph
\renewcommand{\paragraph}[1]{\oldparagraph{#1}\mbox{}}
\fi
\ifx\subparagraph\undefined\else
\let\oldsubparagraph\subparagraph
\renewcommand{\subparagraph}[1]{\oldsubparagraph{#1}\mbox{}}
\fi

%%% Use protect on footnotes to avoid problems with footnotes in titles
\let\rmarkdownfootnote\footnote%
\def\footnote{\protect\rmarkdownfootnote}

%%% Change title format to be more compact
\usepackage{titling}

% Create subtitle command for use in maketitle
\providecommand{\subtitle}[1]{
  \posttitle{
    \begin{center}\large#1\end{center}
    }
}

\setlength{\droptitle}{-2em}

  \title{You Want to Know: What's the Point?}
    \pretitle{\vspace{\droptitle}\centering\huge}
  \posttitle{\par}
    \author{}
    \preauthor{}\postauthor{}
      \predate{\centering\large\emph}
  \postdate{\par}
    \date{2019-06-04}


\begin{document}
\maketitle

The \href{/blog/activity-you-want-to-know/}{You Want to Know activity}
is just one example among many similar activities. This
\textbf{instructors' activity} explores ways you can construct many
similar activities for your students.

\hypertarget{how-to-make-your-own-activity}{%
\subsubsection{How to make your own
activity}\label{how-to-make-your-own-activity}}

The key steps to creating these activities are

\begin{enumerate}
\def\labelenumi{\arabic{enumi}.}
\item
  Determine the learning objectives. These might include

  \begin{enumerate}
  \def\labelenumii{\alph{enumii}.}
  \tightlist
  \item
    Making key distinctions like case/variable,
    categorical/quantitative, explantory/reponse, etc.
  \item
    Identifying key elements of study design.
  \item
    Selecting appropriate analysis methods.
  \item
    Use of software to acheive a task (have students write R command or
    describe how to use other software)
  \end{enumerate}
\item
  Design the questions.

  Depending your goals, you might ask very different things about your
  scenarios.
\item
  Create scenarios that elicit answers targeted at your learning
  objectives.

  Once you have determined the learning objectives and the related
  questions, it can be fun to design scenarios. Pick a theme (how many
  ways can you get donuts into the scenarios?); piggy-back on something
  in the news; etc.
\end{enumerate}

\hypertarget{uses}{%
\subsubsection{Uses}\label{uses}}

These activities can be used in class as group discussions, but they
also make good assessment items for homework, quizzes, or tests. These
scenarios are easy to design at all different stages in a course,
modifying the types of scenarios and the questions asked to correspond
to the learning objectives covered to that point in the course. They can
be used as review or as an introduction to one of the topics involved.

\hypertarget{variations-on-the-theme}{%
\subsubsection{Variations on the Theme}\label{variations-on-the-theme}}

Here's a chance for you to try your hand at creating a ``You Want to
Know'' activity of your own.

\begin{enumerate}
\def\labelenumi{\arabic{enumi}.}
\item
  \textbf{Additional Questions}

  The questions on the example activity sheet are about identifying the
  variables, their types, the parameter(s) of interest, and a plot that
  might be used to explore the data.

  \emph{Come up with some additional questions that might be used with
  these (or other scenarios).}
\end{enumerate}

\begin{enumerate}
\def\labelenumi{\arabic{enumi}.}
\setcounter{enumi}{1}
\item
  \textbf{Digging deeper:} Want to dig deeper on an item? Add a
  follow-up question. Here is an example:

  \begin{itemize}
  \tightlist
  \item
    Example: Choose one of the scenarios that could be designed as a
    randomized experiment. Explain how randomization would be used.
  \end{itemize}

  \emph{Come up with some additional follow-up questions of your own.
  How can you bring multivariable thinking into this activity?}
\end{enumerate}

\begin{enumerate}
\def\labelenumi{\arabic{enumi}.}
\setcounter{enumi}{2}
\item
  \textbf{Different Scenarios}

  \emph{Create additional scenarios.}

  You might like to work backwards from an answer you would like.
  Examples:

  \begin{enumerate}
  \def\labelenumii{\alph{enumii}.}
  \tightlist
  \item
    Create a scenario that has a quantitative response and a categorical
    predictor.
  \item
    Create a scenario that has a categorical response and a quantitative
    predictor.
  \item
    Create a scenario that requires three variables.
  \item
    Create a scenario that should be done using a paired design.
  \item
    Create a scenario that could be analyzed using ANOVA (or some other
    method of your choice.)
  \item
    Create a scenario where side-by-side boxplots might be a good way to
    show the data.
  \item
    Create a scenario that could be done in more than one way. Is one of
    the ways clearly better than the other? Why or why not?
  \end{enumerate}
\end{enumerate}

\hypertarget{be-flexible}{%
\subsubsection{Be flexible}\label{be-flexible}}

For some scenarios, there may be more than one reasonable answer. There
may also be better answers and less good answers, in addition to wrong
answers. If using these activities it is important to help students see
that their can be judgement involved; don't get fixated on the answer
you had in mind. But these can also be used to discuss better an worse
methods for the same goal and why some designs are preferred. (Or why
which is preferable might depend on something not specified by the
scenario.) The important thing is to get students thinking about how to
learn from data.


\end{document}
