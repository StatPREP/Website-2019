\documentclass[]{article}
\usepackage{lmodern}
\usepackage{amssymb,amsmath}
\usepackage{ifxetex,ifluatex}
\usepackage{fixltx2e} % provides \textsubscript
\ifnum 0\ifxetex 1\fi\ifluatex 1\fi=0 % if pdftex
  \usepackage[T1]{fontenc}
  \usepackage[utf8]{inputenc}
\else % if luatex or xelatex
  \ifxetex
    \usepackage{mathspec}
  \else
    \usepackage{fontspec}
  \fi
  \defaultfontfeatures{Ligatures=TeX,Scale=MatchLowercase}
\fi
% use upquote if available, for straight quotes in verbatim environments
\IfFileExists{upquote.sty}{\usepackage{upquote}}{}
% use microtype if available
\IfFileExists{microtype.sty}{%
\usepackage{microtype}
\UseMicrotypeSet[protrusion]{basicmath} % disable protrusion for tt fonts
}{}
\usepackage[margin=1in]{geometry}
\usepackage{hyperref}
\hypersetup{unicode=true,
            pdftitle={You Want to Know: Cases, Variables, and Study Goals},
            pdfborder={0 0 0},
            breaklinks=true}
\urlstyle{same}  % don't use monospace font for urls
\usepackage{longtable,booktabs}
\usepackage{graphicx,grffile}
\makeatletter
\def\maxwidth{\ifdim\Gin@nat@width>\linewidth\linewidth\else\Gin@nat@width\fi}
\def\maxheight{\ifdim\Gin@nat@height>\textheight\textheight\else\Gin@nat@height\fi}
\makeatother
% Scale images if necessary, so that they will not overflow the page
% margins by default, and it is still possible to overwrite the defaults
% using explicit options in \includegraphics[width, height, ...]{}
\setkeys{Gin}{width=\maxwidth,height=\maxheight,keepaspectratio}
\IfFileExists{parskip.sty}{%
\usepackage{parskip}
}{% else
\setlength{\parindent}{0pt}
\setlength{\parskip}{6pt plus 2pt minus 1pt}
}
\setlength{\emergencystretch}{3em}  % prevent overfull lines
\providecommand{\tightlist}{%
  \setlength{\itemsep}{0pt}\setlength{\parskip}{0pt}}
\setcounter{secnumdepth}{0}
% Redefines (sub)paragraphs to behave more like sections
\ifx\paragraph\undefined\else
\let\oldparagraph\paragraph
\renewcommand{\paragraph}[1]{\oldparagraph{#1}\mbox{}}
\fi
\ifx\subparagraph\undefined\else
\let\oldsubparagraph\subparagraph
\renewcommand{\subparagraph}[1]{\oldsubparagraph{#1}\mbox{}}
\fi

%%% Use protect on footnotes to avoid problems with footnotes in titles
\let\rmarkdownfootnote\footnote%
\def\footnote{\protect\rmarkdownfootnote}

%%% Change title format to be more compact
\usepackage{titling}

% Create subtitle command for use in maketitle
\providecommand{\subtitle}[1]{
  \posttitle{
    \begin{center}\large#1\end{center}
    }
}

\setlength{\droptitle}{-2em}

  \title{You Want to Know: Cases, Variables, and Study Goals}
    \pretitle{\vspace{\droptitle}\centering\huge}
  \posttitle{\par}
    \author{}
    \preauthor{}\postauthor{}
      \predate{\centering\large\emph}
  \postdate{\par}
    \date{2019-06-04}


\begin{document}
\maketitle

\pagestyle{empty}

\textbf{Instructions.} In each scenario below,

\begin{itemize}
\item
  determine what variables you would need to collect and whether each is
  categorical or quantitative,
\item
  describe or sketch a graph you would make to give you an initial
  impression regarding the answer to the question of interest,
\item
  determine the parameter(s) of interest and explain how they are
  different from the variables,
\end{itemize}

\begin{itemize}
\tightlist
\item
  determine whether you are using a paired design.
\end{itemize}

Sometimes there may be more than one way to design the study, but don't
design a poor study when a better option is available.

\begin{enumerate}
\def\labelenumi{\arabic{enumi}.}
\item
  You want to know what proportion of students at your institution got a
  flu shot this year.
\item
  You want to know whether male students or female students were more
  likely to get a flu shot this year.
\item
  You want to know whether people who got a flu shot were more likely or
  less likely to get the flu.
\item
  You want to know which of two vaccines is better at preventing flu.
\item
  You want to know whether a new drug works better than an old drug at
  reducing cholesterol.
\item
  You want to know whether rhubarb grows faster or slower if you cover
  it with a bucket for 3 weeks.
\item
  You want to know whether people can swim faster if they wear wet
  suits.
\end{enumerate}

You might find it handy to organize you work into a table like this:

\begin{longtable}[]{@{}lllll@{}}
\toprule
\begin{minipage}[b]{0.11\columnwidth}\raggedright
Scenario\strut
\end{minipage} & \begin{minipage}[b]{0.21\columnwidth}\raggedright
Variables (type)\strut
\end{minipage} & \begin{minipage}[b]{0.17\columnwidth}\raggedright
Parameter(s)\strut
\end{minipage} & \begin{minipage}[b]{0.19\columnwidth}\raggedright
Plot\strut
\end{minipage} & \begin{minipage}[b]{0.17\columnwidth}\raggedright
Paired?\strut
\end{minipage}\tabularnewline
\midrule
\endhead
\begin{minipage}[t]{0.11\columnwidth}\raggedright
1\strut
\end{minipage} & \begin{minipage}[t]{0.21\columnwidth}\raggedright
vaccinated (yes/no) -- categorical\strut
\end{minipage} & \begin{minipage}[t]{0.17\columnwidth}\raggedright
proportion of all students who were vaccinated\strut
\end{minipage} & \begin{minipage}[t]{0.19\columnwidth}\raggedright
bar chart showing number of vaccinated/unvaccinated students\strut
\end{minipage} & \begin{minipage}[t]{0.17\columnwidth}\raggedright
no\strut
\end{minipage}\tabularnewline
\begin{minipage}[t]{0.11\columnwidth}\raggedright
2\strut
\end{minipage} & \begin{minipage}[t]{0.21\columnwidth}\raggedright
\strut
\end{minipage} & \begin{minipage}[t]{0.17\columnwidth}\raggedright
\strut
\end{minipage} & \begin{minipage}[t]{0.19\columnwidth}\raggedright
 \vspace*{0.25in}\strut
\end{minipage} & \begin{minipage}[t]{0.17\columnwidth}\raggedright
\strut
\end{minipage}\tabularnewline
\begin{minipage}[t]{0.11\columnwidth}\raggedright
3\strut
\end{minipage} & \begin{minipage}[t]{0.21\columnwidth}\raggedright
\strut
\end{minipage} & \begin{minipage}[t]{0.17\columnwidth}\raggedright
\strut
\end{minipage} & \begin{minipage}[t]{0.19\columnwidth}\raggedright
 \vspace*{0.25in}\strut
\end{minipage} & \begin{minipage}[t]{0.17\columnwidth}\raggedright
\strut
\end{minipage}\tabularnewline
\begin{minipage}[t]{0.11\columnwidth}\raggedright
4\strut
\end{minipage} & \begin{minipage}[t]{0.21\columnwidth}\raggedright
\strut
\end{minipage} & \begin{minipage}[t]{0.17\columnwidth}\raggedright
\strut
\end{minipage} & \begin{minipage}[t]{0.19\columnwidth}\raggedright
 \vspace*{0.25in}\strut
\end{minipage} & \begin{minipage}[t]{0.17\columnwidth}\raggedright
\strut
\end{minipage}\tabularnewline
\begin{minipage}[t]{0.11\columnwidth}\raggedright
5\strut
\end{minipage} & \begin{minipage}[t]{0.21\columnwidth}\raggedright
\strut
\end{minipage} & \begin{minipage}[t]{0.17\columnwidth}\raggedright
\strut
\end{minipage} & \begin{minipage}[t]{0.19\columnwidth}\raggedright
 \vspace*{0.25in}\strut
\end{minipage} & \begin{minipage}[t]{0.17\columnwidth}\raggedright
\strut
\end{minipage}\tabularnewline
\begin{minipage}[t]{0.11\columnwidth}\raggedright
6\strut
\end{minipage} & \begin{minipage}[t]{0.21\columnwidth}\raggedright
\strut
\end{minipage} & \begin{minipage}[t]{0.17\columnwidth}\raggedright
\strut
\end{minipage} & \begin{minipage}[t]{0.19\columnwidth}\raggedright
 \vspace*{0.25in}\strut
\end{minipage} & \begin{minipage}[t]{0.17\columnwidth}\raggedright
\strut
\end{minipage}\tabularnewline
\begin{minipage}[t]{0.11\columnwidth}\raggedright
7\strut
\end{minipage} & \begin{minipage}[t]{0.21\columnwidth}\raggedright
\strut
\end{minipage} & \begin{minipage}[t]{0.17\columnwidth}\raggedright
\strut
\end{minipage} & \begin{minipage}[t]{0.19\columnwidth}\raggedright
 \vspace*{0.25in}\strut
\end{minipage} & \begin{minipage}[t]{0.17\columnwidth}\raggedright
\strut
\end{minipage}\tabularnewline
\bottomrule
\end{longtable}


\end{document}
