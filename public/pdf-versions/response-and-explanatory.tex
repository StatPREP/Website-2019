\documentclass[nofonts,]{tufte-handout}

% ams
\usepackage{amssymb,amsmath}

\usepackage{ifxetex,ifluatex}
\usepackage{fixltx2e} % provides \textsubscript
\ifnum 0\ifxetex 1\fi\ifluatex 1\fi=0 % if pdftex
  \usepackage[T1]{fontenc}
  \usepackage[utf8]{inputenc}
\else % if luatex or xelatex
  \makeatletter
  \@ifpackageloaded{fontspec}{}{\usepackage{fontspec}}
  \makeatother
  \defaultfontfeatures{Ligatures=TeX,Scale=MatchLowercase}
  \makeatletter
  \@ifpackageloaded{soul}{
     \renewcommand\allcapsspacing[1]{{\addfontfeature{LetterSpace=15}#1}}
     \renewcommand\smallcapsspacing[1]{{\addfontfeature{LetterSpace=10}#1}}
   }{}
  \makeatother
\fi

% graphix
\usepackage{graphicx}
\setkeys{Gin}{width=\linewidth,totalheight=\textheight,keepaspectratio}

% booktabs
\usepackage{booktabs}

% url
\usepackage{url}

% hyperref
\usepackage{hyperref}

% units.
\usepackage{units}


\setcounter{secnumdepth}{-1}

% citations
\usepackage{natbib}
\bibliographystyle{plainnat}

%% tint override
\setcitestyle{round} 

% pandoc syntax highlighting

% longtable
\usepackage{longtable,booktabs}

% multiplecol
\usepackage{multicol}

% strikeout
\usepackage[normalem]{ulem}

% morefloats
\usepackage{morefloats}


% tightlist macro required by pandoc >= 1.14
\providecommand{\tightlist}{%
  \setlength{\itemsep}{0pt}\setlength{\parskip}{0pt}}

% title / author / date
\title{Response and explanatory variables}
\date{StatPREP Class Lesson}

%% -- tint overrides
%% fonts, using roboto (condensed) as default
\usepackage[sfdefault,condensed]{roboto}
%% also nice: \usepackage[default]{lato}

%% colored links, setting 'borrowed' from RJournal.sty with 'Thanks, Achim!'
\RequirePackage{color}
\definecolor{link}{rgb}{0.1,0.1,0.8} %% blue with some grey
\hypersetup{
  colorlinks,%
  citecolor=link,%
  filecolor=link,%
  linkcolor=link,%
  urlcolor=link
}

%% macros
\makeatletter

%% -- tint does not use italics or allcaps in title
\renewcommand{\maketitle}{%     
  \newpage
  \global\@topnum\z@% prevent floats from being placed at the top of the page
  \begingroup
    \setlength{\parindent}{0pt}%
    \setlength{\parskip}{4pt}%
    \let\@@title\@empty
    \let\@@author\@empty
    \let\@@date\@empty
    \ifthenelse{\boolean{@tufte@sfsidenotes}}{%
      %\gdef\@@title{\sffamily\LARGE\allcaps{\@title}\par}%
      %\gdef\@@author{\sffamily\Large\allcaps{\@author}\par}%
      %\gdef\@@date{\sffamily\Large\allcaps{\@date}\par}%
      \gdef\@@title{\begingroup\fontseries{b}\selectfont\LARGE{\@title}\par}%
      \gdef\@@author{\begingroup\fontseries{l}\selectfont\Large{\@author}\par}%
      \gdef\@@date{\begingroup\fontseries{l}\selectfont\Large{\@date}\par}%
    }{%
      %\gdef\@@title{\LARGE\itshape\@title\par}%
      %\gdef\@@author{\Large\itshape\@author\par}%
      %\gdef\@@date{\Large\itshape\@date\par}%
      \gdef\@@title{\begingroup\fontseries{b}\selectfont\LARGE\@title\par\endgroup}%
      \gdef\@@author{\begingroup\fontseries{l}\selectfont\Large\@author\par\endgroup}%
      \gdef\@@date{\begingroup\fontseries{l}\selectfont\Large\@date\par\endgroup}%
    }%
    \@@title
    \@@author
    \@@date
  \endgroup
  \thispagestyle{plain}% suppress the running head
  \tuftebreak% add some space before the text begins
  \@afterindentfalse\@afterheading% suppress indentation of the next paragraph
}

%% -- tint does not use italics or allcaps in section/subsection/paragraph
\titleformat{\section}%
  [hang]% shape
  %{\normalfont\Large\itshape}% format applied to label+text
  {\fontseries{b}\selectfont\Large}% format applied to label+text
  {\thesection}% label
  {1em}% horizontal separation between label and title body
  {}% before the title body
  []% after the title body

\titleformat{\subsection}%
  [hang]% shape
  %{\normalfont\large\itshape}% format applied to label+text
  {\fontseries{m}\selectfont\large}% format applied to label+text
  {\thesubsection}% label
  {1em}% horizontal separation between label and title body
  {}% before the title body
  []% after the title body

\titleformat{\paragraph}%
  [runin]% shape
  %{\normalfont\itshape}% format applied to label+text
  {\fontseries{l}\selectfont}% format applied to label+text
  {\theparagraph}% label
  {1em}% horizontal separation between label and title body
  {}% before the title body
  []% after the title body

%% -- tint does not use italics here either
% Formatting for main TOC (printed in front matter)
% {section} [left] {above} {before w/label} {before w/o label} {filler + page} [after]
\ifthenelse{\boolean{@tufte@toc}}{%
  \titlecontents{part}% FIXME
    [0em] % distance from left margin
    %{\vspace{1.5\baselineskip}\begin{fullwidth}\LARGE\rmfamily\itshape} % above (global formatting of entry)
    {\vspace{1.5\baselineskip}\begin{fullwidth}\fontseries{m}\selectfont\LARGE} % above (global formatting of entry)
    {\contentslabel{2em}} % before w/label (label = ``II'')
    {} % before w/o label
    {\rmfamily\upshape\qquad\thecontentspage} % filler + page (leaders and page num)
    [\end{fullwidth}] % after
  \titlecontents{chapter}%
    [0em] % distance from left margin
    %{\vspace{1.5\baselineskip}\begin{fullwidth}\LARGE\rmfamily\itshape} % above (global formatting of entry)
    {\vspace{1.5\baselineskip}\begin{fullwidth}\fontseries{m}\selectfont\LARGE} % above (global formatting of entry)
    {\hspace*{0em}\contentslabel{2em}} % before w/label (label = ``2'')
    {\hspace*{0em}} % before w/o label
    %{\rmfamily\upshape\qquad\thecontentspage} % filler + page (leaders and page num)
    {\upshape\qquad\thecontentspage} % filler + page (leaders and page num)
    [\end{fullwidth}] % after
  \titlecontents{section}% FIXME
    [0em] % distance from left margin
    %{\vspace{0\baselineskip}\begin{fullwidth}\Large\rmfamily\itshape} % above (global formatting of entry)
    {\vspace{0\baselineskip}\begin{fullwidth}\fontseries{m}\selectfont\Large} % above (global formatting of entry)
    {\hspace*{2em}\contentslabel{2em}} % before w/label (label = ``2.6'')
    {\hspace*{2em}} % before w/o label
    %{\rmfamily\upshape\qquad\thecontentspage} % filler + page (leaders and page num)
    {\upshape\qquad\thecontentspage} % filler + page (leaders and page num)
    [\end{fullwidth}] % after
  \titlecontents{subsection}% FIXME
    [0em] % distance from left margin
    %{\vspace{0\baselineskip}\begin{fullwidth}\large\rmfamily\itshape} % above (global formatting of entry)
    {\vspace{0\baselineskip}\begin{fullwidth}\fontseries{m}\selectfont\large} % above (global formatting of entry)
    {\hspace*{4em}\contentslabel{4em}} % before w/label (label = ``2.6.1'')
    {\hspace*{4em}} % before w/o label
    %{\rmfamily\upshape\qquad\thecontentspage} % filler + page (leaders and page num)
    {\upshape\qquad\thecontentspage} % filler + page (leaders and page num)
    [\end{fullwidth}] % after
  \titlecontents{paragraph}% FIXME
    [0em] % distance from left margin
    %{\vspace{0\baselineskip}\begin{fullwidth}\normalsize\rmfamily\itshape} % above (global formatting of entry)
    {\vspace{0\baselineskip}\begin{fullwidth}\fontseries{m}\selectfont\normalsize\rmfamily} % above (global formatting of entry)
    {\hspace*{6em}\contentslabel{2em}} % before w/label (label = ``2.6.0.0.1'')
    {\hspace*{6em}} % before w/o label
    %{\rmfamily\upshape\qquad\thecontentspage} % filler + page (leaders and page num)
    {\upshape\qquad\thecontentspage} % filler + page (leaders and page num)
    [\end{fullwidth}] % after
}{}

  
\makeatother



\begin{document}

\maketitle




\hypertarget{orientation}{%
\section{Orientation}\label{orientation}}

A variable is a quantity or characteristic that varies from one person
to another, or more generally, from one \emph{unit of observation} to
another. There are two distinct \emph{types} of variables: quantitative
(numeric) and categorical (labels/words/etc.).

There's another distinction to be made about variables, which is not
about the type of the variable itself but the \emph{role} the variable
will play in the statistical methods we use to describe a situation of
interest and the relationships among the variables involved. This
distinction is between the \emph{response variable} and the
\emph{explanatory variables}.

Think about how we describe human relationships. For example, consider
two women: Eliana and Rabia. One possible relationship: Eliana is
Rabia's aunt. Another possible relationship: Rabia is Eliana's niece. As
you know, these two relationships are exactly the same thing, just
expressed differently. Each of the expressions involves a reference
person, that is, a person with respect to whom the relationship word
(``aunt'', ``niece'') is used.

Similarly, when we describe a relationship between two variables, it's
helpful to consider one variable to be the reference. We describe the
pattern of the other variable \emph{with respect to} the first variable.
Now phrases like ``one variable,'' ``other variable,'' and ``first
variable'' are ambiguous. It's hard to keep track of which variable is
which. So, to simplify things, we identify one of the variables as the
\emph{response variable} and the other as the \emph{explanatory
variable}. Just as the two women can be related by either ``aunt'' or
``niece'', the roles of the two variables can be described as ``Response
is a \emph{function} of Explanatory'' or ``Explanatory \emph{determines}
Response.''

How to decide which variable should play the role of the response and
which the explanatory? There's no absolute rule; generally you can
describe relationships either way, keeping in mind that the statement of
the relationship (``aunt'', ``niece'') will depend on which is which.
Here are some rules of thumb.

\begin{itemize}
\tightlist
\item
  If you are trying to \emph{predict} the value of one variable by
  another, call the variable to be predicted the \emph{response}
  variable. Example: Price of a rental apartment as a function of the
  city in which it's located.
\item
  If you believe that one variable \emph{causes} the other variable,
  call the variable being caused the \emph{response} variable. Example:
  Risk of cervical cancer as explained by whether or not the person got
  a human papilloma virus (HPV) vaccine.
\item
  If there's an \emph{outcome} you're interested in, make that the
  \emph{response} variable. Example: Test scores as explained by
  socio-economic status.
\end{itemize}

Most statistics books have traditionally been written without making
much use of the concepts of response and explanatory variables.
Mathematicians, in particular, seem to like to talk about ``bivariate
relationships'' and to use statistics like the correlation coefficient
that are the same whichever order you put the variables in. But most
two-variable statistics -- for instance the slope of a regression line,
or the difference between two proportions -- depend on the order of the
variables. As such, it's helpful to explicit in naming the particular
variable you want to predict or that's being caused, or that you have a
particular interest in.

And, when there are more than two variables, it's particularly helpful
to distinguish between the single response variable and the multiple
explanatory variables (which are sometimes called \emph{covariates} or
\emph{confounders}.)

\hypertarget{activity}{%
\section{Activity}\label{activity}}

There are two main types of variables: quantitative and categorical.
Strictly as a matter of logic, there are four possible ways that these
two types can be arranged as the response and explanatory variables.
It's important to know this, since the choice of an appropriate
statistical technique should be shaped by the types of the response and
explanatory variables.

\begin{longtable}[]{@{}llll@{}}
\toprule
Response & Explanatory & Statistical technique & Little
App\tabularnewline
\midrule
\endhead
Quantitative & Quantitative & linear regression &
\href{https://dtkaplan.shinyapps.io/LA_linear_regression/}{linear
regression}\footnote{\url{https://dtkaplan.shinyapps.io/LA_linear_regression/}}.\tabularnewline
Quantitative & Categorical & group-wise means &
\href{https://dtkaplan.shinyapps.io/LA_t_test/}{t-test}\footnote{\url{https://dtkaplan.shinyapps.io/LA_t_test/}}.\tabularnewline
Categorical & Quantitative & proportion regression &
\href{https://dtkaplan.shinyapps.io/LA_proportions/}{proportions}\footnote{\url{https://dtkaplan.shinyapps.io/LA_proportions/}}.\tabularnewline
Categorical & Categorical & group-wise proportions &
\href{https://dtkaplan.shinyapps.io/LA_proportions/}{proportions}\footnote{\url{https://dtkaplan.shinyapps.io/LA_proportions/}}.\tabularnewline
\bottomrule
\end{longtable}

Depending on where you are in you statistics course, you might have not
yet encountered one of these techniques or the other.

Here are a few pairs of variables:

\hypertarget{some-pairs-of-variables}{%
\subsection{Some pairs of variables}\label{some-pairs-of-variables}}

These variables are found in the data available through the Little App.
For simplicity, we're using only those categorical variables that have
just two levels.

\begin{itemize}
\tightlist
\item
  \texttt{Births\_2014}

  \begin{enumerate}
  \def\labelenumi{\arabic{enumi}.}
  \tightlist
  \item
    Whether the mother is covered by the WIC program and the age of
    mother.
  \item
    The age of the mother and of the father.
  \item
    The length of gestation and the baby's weight at birth.
  \end{enumerate}
\item
  \texttt{NHANES}

  \begin{enumerate}
  \def\labelenumi{\arabic{enumi}.}
  \setcounter{enumi}{3}
  \tightlist
  \item
    Systolic blood pressure and sex.
  \item
    Diabetes and age
  \item
    Weight and body mass index (BMI)
  \item
    Income and \texttt{home\_type}
  \end{enumerate}
\item
  \texttt{diamonds}

  \begin{enumerate}
  \def\labelenumi{\arabic{enumi}.}
  \setcounter{enumi}{7}
  \tightlist
  \item
    The price of a diamond and its weight in carats.
  \item
    The weight of a diamond and its clarity.
  \end{enumerate}
\end{itemize}

Answer the following questions for each of the 9 pairs of variables
listed.

\begin{enumerate}
\def\labelenumi{\arabic{enumi}.}
\item
  View the data with a point plot, using the
  \href{https://dtkaplan.shinyapps.io/LA_point_plot/}{point-plot Little
  App}. This will let you readily see which type each variable is. (The
  choice of one variable as response and one as explanatory isn't
  critical here, since the reverse choice would generate the same plot
  but turned around the diagonal to reverse the axes.)

  Use the point-plot Little App to visualize the relationship.
  Characterize it as ``strong'', ``moderate'', ``weak'' or ``none''. You
  don't need to give a more detailed description. We just want you to
  decide whether you can see a clear relationship or not.

  \emph{For each of the nine variable pairs, characterize the
  relationship as `strong', `moderate', and so on.} ~~.~~.~~. ~

  ~

  ~

  ~

  ~
\item
  In order to give a detailed description of the relationship, or
  sometimes in order to detect any relationship at all, you need to
  calculate appropriate statistics on the data. For this, you have to
  pick the appropriate Little App, which often means that you need to
  designate one variable as the response and the other one as the
  explanatory variable. (You can refer to the table above.)

  For each of the nine pairs of variables, make an appropriate choice of
  response and explanatory variables using the rules of thumb above.
  Since some of these depend on the shape of your particular interest in
  the variables, your choice might be different from a classmate's.

  \emph{For each of the nine pairs of variables, write down your choices
  for response and explanatory variables here, along with the
  appropriate LittleApp from the table above.} ~~.~~.~~. ~

  ~

  ~

  ~

  ~
\item
  Now you are going to try to describe the relationships using
  statistics. Just like a forest can be obscured by the trees, so a
  relationship can be obscured by the scatter of data. Often, the
  appropriate statistic makes clearer the relationship.

  Open up the appropriate Little App from step (3) and turn on whatever
  statistics you can use to describe the relationship. (You can choose
  the sample size to be whatever you want, but generally a larger sample
  size makes it easier to see a relationship. You may also choose to
  \emph{stratify} the sample.)

  Give an appropriate English-language description of the relationship.
  Some of the terms you might use are ``upward sloping function,''
  ``downward sloping function'', ``difference of means,'' ``difference
  of proportions,'' ``no relationship.''

  \emph{For each of the nine pairs of variables, write down your
  English-language description of the relationship.} ~~.~~.~~. ~

  ~

  ~

  ~

  ~
\end{enumerate}

\begin{center}\rule{0.5\linewidth}{\linethickness}\end{center}

Version 0.3, 2019-05-28, Danny Kaplan,



\end{document}
